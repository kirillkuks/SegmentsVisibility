\documentclass[letterpaper,12pt]{article}
\usepackage[utf8]{inputenc}
\usepackage[russian]{babel}
\usepackage[left=2cm,right=2cm,top=2cm,bottom=2cm,bindingoffset=0cm]{geometry}
\usepackage{graphicx}
\graphicspath{{images/}}
\usepackage{float}
\usepackage{wrapfig}

\begin{document}

\begin{center}
      Алгоритм определения правильности отрезка.
\end{center}

Входные данные: отрезок $s$, грань РСДС $f$, упорядоченное по 
полярному углу множество точек $P_u$, индексы $i, j$ краевых точек $s$
в $P_u$.

Выходные данные: <<да>>, если $s$ правильный. <<нет>>, если $s$ 
неправильный.

\begin{center}
Определение.
Положением точки в упорядоченном (цикличном) массиве будем называть пару соседних точек
(левую и правую). Будем говорить, что точка $p$ поменяла положение в упорядоченном массиве,
если $p$ поменялся хотя бы один сосед.
\end{center}

Алгоритм: если точки рядом, то ответ <<нет>>. Иначе
выбираем произвольную внутреннюю точку $f$, находим ориентированную
площадь треугольника $q p_i p_j$ (Не умаляя общности, считаем что
точка $p_i$ раньше $p_j$ в упорядоченном списке), если она положительна,
то ответ <<да>>, иначе ответ <<нет>>.

\hspace{4em}

\begin{center}
      Теорема. Статус не зависит от выбора точки внутри $f$
\end{center}
Доказательство.
Статус - упорядоченные по полярному углу концы отрезков, правильность каждого 
отрезка.
\begin{enumerate}
      \item Упорядочение не поменяется.
            Предположим противное, пусть в $f$ существует две точки $A_1, A_2$, 
            для которых порядок в упорядоченном массиве отличается. Не умаляя 
            общности, будем считать, что порядок отличается только для двух
            точек $p_1, p_2$. Тогда в силу непрерывности значения полярного 
            угла каждой точки  $\exists t \in (0, 1) : A(t) = (1 - t)A_1 + tA_2$ 
            такая, что точки $p_1, p_2$ имеют одинаковый полярный угол 
            относительно $ A(t) $. Получили противоречие. 
      \item Правильность отрезков.
            Заметим, что правильность отрезка может поменяться, только в двух 
            случаях:
            \begin{itemize}
                  \item Точки поменялись местами в упорядоченном списке. Что 
                        невозможно по первому пункту.
                  \item Поменялся поворот угла, что возможно только при 
                        перешагивании ребра, построенного на точках одного 
                        отрезка.
            \end{itemize}
\end{enumerate}

\hspace{4em}

\begin{center}
      Теорема. В процессе перешагивания ребра статус меняется не сильно.
\end{center}
Доказательство.
Статус - упорядоченные по полярному углу концы отрезков, 
правильность каждого отрезка. Обозначим ребро за $e$, грани, которые оно 
разделяет за $f_1$ и $f_2$.
\begin{enumerate}
      \item Упорядочение поменяется только для не более чем двух точек.
 
            Предположим противное, это означает, что
            $\forall \varepsilon > 0 : \exists A_1 \in f_1, A_2 \in f_2 : 
            \exists A_0 \in e  :  |A_0 - A_1|  < \varepsilon, 
            |A_0 - A_2|  < \varepsilon$,
            кроме того, порядок точек в статусе в $A_1$, $A_2$ различен 
            для $n > 2$ точек. В силу бесконечной малости отрезка $A_1 A_2$,
            все $n$ точек должны иметь одинаковые полярные углы в точке,
            лежащей на $e$, а значит на прямой, порожденной $e$ лежит $>2$
            точек, что противоречит условию МТОП.
      \item Правильность отрезков.

            При перешагивании ребра $e$ 
            в упорядоченном массиве для каких-то точек поменялись соседи.
            По первому пункту таких точек не более чем $4$.
            Также для максимум двух отрезков (на концах которых построено ребро $e$)
            может поменяться поворот.
            Покажем, что для $n-4$ отрезков, концы которых не поменяли соседей,
            их правильность (и неправильность) не изменились.
            \begin{itemize}
                  \item Отрезок был правильным.

                  Так как концы этого отрезка не поменяли соседей,
                  то они также останутся соседними в упорядоченном массиве,
                  тогда отрезок может стать неправильным, только если поменяется поворот.
                  Поворот отрезка может поменяться только,
                  если $e$ построено но концах одного отрезка,
                  чего не может быть по выбору $n-4$ отрезков.
                  Значит, рассматриваемый отрезок останется правильным. 

                  \item Отрезок был неправильным, потому что точки не были соседями.
                  
                  Так как концы этого отрезка не поменяли соседей,
                  то они также останутся не соседними в упорядоченном массиве.
                  Тогда и сам отрезок останется неправильным.

                  \item Отрезок был неправильным, потому что имел правый поворот.

                  Поворот отрезка может поменяться только,
                  если $e$ построено но концах одного отрезка,
                  чего не может быть по выбору $n-4$ отрезков.
                  Значит, рассматриваемый отрезок останется неправильным. 

            \end{itemize}
\end{enumerate}
\end{document}

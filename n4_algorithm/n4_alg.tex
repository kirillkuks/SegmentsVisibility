\documentclass[letterpaper,12pt]{article}
\usepackage[utf8]{inputenc}
\usepackage[russian]{babel}
\usepackage[left=2cm,right=2cm,top=2cm,bottom=2cm,bindingoffset=0cm]{geometry}

\begin{document}
\begin{center}
Описание алгоритма за $O(n^4)$
\end{center}
\par
Алгоритм принимает множество $S$ из $n$ непересекающихся отрезков.
$P$ - множесто концов отрезков $S$, является МТОП-1.
\begin{enumerate}
      \item Строится множество $L$, прямых, определяющихся любой парой 
            $p_i, p_j \in P$.
            \par
            Данная операция требует $O(n^2)$ времени, $|L| = 2n^2-n$.
            \par
            Для дальнейшей обработки множество точек $P$ сохраняется.
            Для каждого отрезка запоминаются индексы его концов.
            Контейнер занимает $2n$ памяти.
      \item Строится упорядочение прямых $A(L)$ с помощью 
            инкрементального алгоритма. Его сложность для $m$ прямых
            есть $O(m^2)$. Соответственно в данном случае потребуется
            $O(n^4)$.
            \par
            Результатом работы данного алгоритма является РСДС,
            занимающий линейную память от числа ребер, вершин и граней.
            Так как это число для $A(L)$ квадратично зависит от количества
            прямых в $L$, то расход по памяти на данном этапе алгоритма 
            достигает $O(n^4)$.
            \par
            Для каждого ребра сохраняется информация об отрезках из $S$,
            на точках которых была построена прямая, 
            частью которой является данное ребро.
      \item Произвольно выбирается грань $f$ РСДС. Производится 
            упорядочивание $P$ по возрастанию полярного угла относительно 
            любой внутренней точки $f$.
            \par
            Производится заметание по углу, в процессе которого получаем
            кольцевой список, хранящий следующие данные:
            \begin{enumerate}
                  \item Индексы точек, задающих сектор. 
                  \item Статус, содержащий информацию о нахождении в секторе
                        для каждого отрезка (bool контейнер размера $n$)
            \end{enumerate}
            \par
            Статус для нулевого сектора находится честной проверкой на 
            пересечение с помощью построения бесконечной прямой по нулевому
            углу. Статусы остальных секторов определяются за константу,
            исходя из события: меняем знак в статусе того отрезка, чья
            точка - новое событие. Также в процессе заметания 
            подсчитывается и запоминается число секторов ($k$), у которых
            в статусе $>1$ отрезка (со значенем $True$). 
            \par
            Сортировка занимает $O(nlog(n))$, заметание - $O(n)$.
      \item Производится обход РСДС (например в ширину), начиная с грани $f$.
            На каждом шаге, зная <<перешагиваемое>> ребро, через обращение к 
            отрезкам находятся индексы $i$ и $j$, порождающиx его точек.
            Зная точки, можем найти секторы, которые оказываются затронутыми
            на этом шаге.
            Изменения, которые претерпевает список определяется конечным
            набором возможных случаев. Все случаи отдельно рассмотрены в
            приложении.
            \par
            Обработка каждой грани требует $O(1)$ времени и памяти. 
      \item Обход продолжается до тех пор пока не найдется грань, 
            для которой после пересчета статусов число $k$ примет значение $0$,
            или пока не останется непосещенных граней. В первом случае ответом 
            алгоритма - <<да>> с предоставлением любой точки внутри $f$, 
            во втором случае ответ - <<нет>>.
            \par
            Обход требует $O(N^4)$ времени.
\end{enumerate}
\end{document}
\documentclass[letterpaper,12pt]{article}
\usepackage[utf8]{inputenc}
\usepackage[russian]{babel}
\usepackage[left=2cm,right=2cm,top=2cm,bottom=2cm,bindingoffset=0cm]{geometry}
\usepackage{hyperref}

\hypersetup{
    colorlinks=true,
    linkcolor=blue,
    }

\begin{document}
\begin{center}
Описание алгоритма за $O(n^4)$
\end{center}
\par
Алгоритм принимает множество $S$ из $n$ непересекающихся отрезков.
$P$ - множесто концов отрезков $S$, является МТОП.
\begin{enumerate}
      \item Строится множество $L$, прямых, определяющихся любой парой 
            $p_i, p_j \in P$.
            \par
            Данная операция требует $O(n^2)$ времени, $|L| = 2n^2-n$.
            \par
            Для дальнейшей обработки множество точек $P$ сохраняется.
            Для каждого отрезка запоминаются индексы его концов.
            Контейнер занимает $2n$ памяти.
      \item Строится упорядочение прямых $A(L)$ с помощью 
            инкрементального алгоритма. Его сложность для $m$ прямых есть 
            \hyperlink{literature_1}{$O(m^2)$}. Соответственно в данном 
            случае потребуется $O(n^4)$.
            \par
            Результатом работы данного алгоритма является РСДС,
            занимающий линейную память от числа ребер, вершин и граней.
            Так как это число для $A(L)$ квадратично зависит от количества
            прямых в $L$, то расход по памяти на данном этапе алгоритма 
            достигает $O(n^4)$.
            \par
            Для каждого ребра сохраняется информация об отрезках из $S$,
            на точках которых была построена прямая, частью которой 
            является данное ребро.
      \item Произвольно выбирается грань $f$ РСДС. Производится 
            упорядочивание $P$ по возрастанию полярного угла относительно 
            любой внутренней точки грани, обозначим точку $q$.
            \par
            После сортировки все отрезки проверяются на <<правильность>>.
            (индексы $i$ и $j$ точек отрезка в контейнере должны быть соседними)
            Запоминается количество правильных отрезков. Сортировка занимает 
            $O(nlog(n))$, проверка на правильность - $O(n)$.
      \item Производится обход РСДС (например в ширину), начиная с грани $f$. 
            На каждом шаге, зная <<перешагиваемое>> ребро, через обращение к 
            отрезкам находятся индексы $i$ и $j$, порождающиx его точек. 
            \par
            Если на следующем шаге оказывается необходимым перешагнуть через
            ребро, которое является частью одного из исходных отрезков, то
            данный шаг пропускается. Определение этого факта занимает 
            константное время.
            \par
            Все грани, которые имеют общей точкой - одну из точек отрезка, образуют цикл, 
            поэтому выбрасывание из графа обхода ребра, упомянутого выше, 
            никак не влияет на связность самого графа.
            \par
            Существует конечное количество случаев того, как может изменится
            сектор заметания, заданный точками $p_i, p_j$ после <<перешагивания>>.
            Все случаи подробно рассмотрены в приложении.
            \par
            С учетом имеющихся данных, случай определяется за 
            константное время. Столько же требуется на внесение изменений 
            в порядок точек в контейнере, в количество <<правильных>>
            отрезков.       
            Таким образом, обработка каждой грани требует $O(1)$ времени и памяти. 
      \item Обход продолжается до тех пор пока не найдется грань, 
            в которой все отрезки окажутся <<правильными>>, или пока не
            останется непосещенных граней. В первом случае ответом 
            алгоритма - <<да>> с предоставлением любой точки внутри 
            найденной грани, во втором случае ответ - <<нет>>.
            \par
            Обход требует $O(N^4)$ времени.
\end{enumerate}
\par
Источники:
\par
\hypertarget{literature_1}{1.} de Berg, Mark; Cheong, Otfried; van Kreveld, Marc; Overmars, Mark (2008). 
Computational Geometry, Algorithms and Applications (3rd ed.). 
Springer. pp. 172–177. ISBN 978-3-540-77973-5.
\end{document}